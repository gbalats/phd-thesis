\label{correctness}

We shall call the path-edges originating from known-nodes
\emph{\emph{kp}-edges}.  We will also use the symbols $S_0,
S_1,\ldots,S_\infty$ to denote the various stratifications computed at
each step of Algorithm~\ref{alg:multiple:strat}.  Note that our
algorithm will actually produce a finite number of stratifications (at
most $|V|$) but we can disregard both the upper limit of the main loop
and the early-failure condition (line~\ref{lst:line:unsat}) for
proving correctness. Instead we focus on the main computation
(line~\ref{lst:line:formula}) and the infinite sequence of
stratifications that would be produced if it was allowed to run
forever (even after reaching a fixpoint).

\begin{lem}\label{monotonicity}
  %% Let $G = (V,E) : V = V_{known} \cup V_{phantom}$ and $E = E_{path}
  %% \cup E_{direct}$ be a constraint graph.
  For all $v \in V$, the sequence $\{ S_{0}[v], S_{1}[v], \ldots \}$
  is non-decreasing.
\end{lem}
\begin{proof}
  Direct consequence of line~\ref{lst:line:formula} of the algorithm.
\end{proof}

\begin{lem}\label{range}
  %% Let $G = (V,E) : V = V_{known} \cup V_{phantom}$ and $E = E_{path}
  %% \cup E_{direct}$ be a constraint graph.
  For all $i \in \mathbb{N}$, $0 \leq S_i[v] \leq i, \forall v \in V$.
\end{lem}

\begin{proof}
  Induction on step $i$.

  \begin{enumerate}
  \item \emph{Base}: For $i = 0$, we have that $S_i[v] = S_0[v] = 0, \forall v
    \in V$.
  \item \emph{Inductive Step}: Assume that $0 \leq S_n[v] \leq n,
    \forall v \in V$ for some value of $n$. We must show that $0 \leq
    S_{n+1}[v] \leq n + 1, \forall v \in V$. Let $k$ be a node in $V$.
    Either $S_{n+1}[k] = S_{n}[k]$, and therefore $0 \leq S_{n+1}[k] \leq
    n$, or there will exist a node $s$, s.t. $S_{n+1}[k] = S_{n}[s] +
    1$, in which case $1 \leq S_{n+1}[k] \leq n + 1$.
  \end{enumerate}
\end{proof}

\begin{defn}
  A node $v \in V$ is \emph{$i$-stabilized} if and only if $S_i[v] =
  S_{i+1}[v]$ and either $i = 0$ or $S_{i-1}[v] < S_i[v]$.
\end{defn}

\begin{thm}\label{stability} (Once a node's stratum does not change,
  it will not change again.)
  %% Let $G = (V,E) : V = V_{known} \cup V_{phantom}$ and $E = E_{path}
  %% \cup E_{direct}$ be a constraint graph.
  If $S_{i}[v] = S_{i+1}[v]$ for some node $v \in V$ and a value $i
  \in \mathbb{N}$, then $S_{j}[v] = S_{i}[v], \forall j \in
  \mathbb{N}$ such that $j \geq i$.
\end{thm}

\begin{proof}
  Induction on step $i$.

  \begin{enumerate}
  \item \emph{Base}: For $i = 0$, let $v$ be a node in $V$, such that
    $S_0[v] = S_{1}[v]$. From Lemma~\ref{range}, we have that $S_0[v]
    = S_{1}[v] = 0$, which can happen if and only if $v$ has no
    incoming edges (otherwise an edge would cause the node to move to
    a higher stratum on iteration 1). It is therefore impossible for
    $v$ to change in the following iterations since it has no
    constraining edges.
  \item \emph{Inductive Step}: Assume that the theorem holds for all $i
    < n$ for some value of $n \in \mathbb{N}$. Let $t \in V$ be a
    node, such that $S_n[t] = S_{n+1}[t]$.  We will show that $t$'s stratum will
    not change in the future. It suffices to prove that, for each of
    $t$'s constraining edges, there will be a node $s$ that has
    already been stabilized at a lower stratum than $t$, and can be
    used to satisfy the constraint at this point. Therefore, the
    constraint will remain satisfied in future iterations due to $s$,
    which will remain in the same stratum from now on (induction
    hypothesis). For ordinary \emph{path}-edges, node $s$ is no other
    than the source of the edge itself, while for \emph{kp}-edges,
    it is the lower phantom projection of the edge's source at step
    $n$ that we may use instead. Let us consider ordinary path-edges
    first, in more detail. From Lemma~\ref{range}, we have that $0
    \leq S_n[t] \leq n$, and thus $0 \leq S_{n+1}[t] \leq n$.  Let
    $(s,t) \in E$ be an incoming edge of $t$. We have that $0\leq
    S_{n}[s] < S_{n+1}[t] \leq n$ which entails that $0 \leq S_{n}[s]
    \leq n - 1$.  Therefore, according to Lemma~\ref{range}, we have
    that $0 \leq S_{i}[s] \leq n - 1, \forall i \in \{0,\ldots{},n\}$.
    By the pigeonhole principle, and due to Lemma~\ref{monotonicity},
    there must surely exist two consecutive values $i,i+1$,
    s.t. $S_i[s] = S_{i+1}[s]$ and $i < n$. From the induction
    hypothesis, we know that $s$ will therefore not change and its
    constraint on $t$ will be irrelevant in future iterations. We
    proceed similarly, for a \emph{kp}-edge $(s,t)$ (where we use
    the lowest phantom projection of $s$ at this point, instead of $s$
    itself). Thus, $t$ will not change in the future, since every
    constraint of $t$ will remain satisfied after this iteration.
  \end{enumerate}
\end{proof}

\begin{corollary}\label{maxstrata}
  %% Let $G = (V,E) : V = V_{known} \cup V_{phantom}$ and $E = E_{path}
  %% \cup E_{direct}$ be a constraint graph.
  For all $v \in V$ and $n \in \mathbb{N}^+$, $S_{n-1}[v] \neq
  S_{n}[v] \Rightarrow S_{n}[v] = n$.
\end{corollary}

\begin{thm}\label{termination}
  %% Let $G = (V,E) : V = V_{known} \cup V_{phantom}$ and $E = E_{path}
  %% \cup E_{direct}$ be a constraint graph.
  The stratification sequence $S_0, S_1,\ldots$ will diverge (i.e.,
  not reach a
  fixpoint) if and only if at some computation step, $n$, no new nodes stabilize
  and not all nodes have
  already stabilized---that is, $\exists n
  \in \mathbb{N}^+$, such that: for some $v \in V$, $S_{n+1}[v] \neq
  S_{n}[v]$ and for all $v \in V$, $S_{n+1}[v] = S_{n}[v] \Rightarrow
  S_{n}[v] = S_{n-1}[v]$.
\end{thm}

\begin{proof}\hspace*{\fill}
  \begin{enumerate}
  \item (\emph{If}) Let $n$ be a computation step, such
    that $(\forall v \in V) \; (S_{n-1}[v] \neq S_{n}[v] \Rightarrow
    S_{n}[v] \neq S_{n+1}[v])$. We can disregard any node $u$ such
    that $S_{n-1}[u] = S_{n}[u]$, and observe that for each remaining
    node, there must exist at least a constraining edge that cannot be
    satisfied with a node that has already been ``stabilized''. That
    said, due to Corollary~\ref{maxstrata}, each remaining node $v \in
    V$, s.t. $S_{n-1}[v] \neq S_{n}[v]$, will be placed at a higher (by 1)
    stratum at this point, i.e., $S_{i}[v] = i, \forall i \in
    \{0,\ldots{},n+1\}$. Since the relative positions of all the
    remaining nodes will be the same at step $n+1$ as they had been at
    step $n$, there is no way for a node to be stabilized at this last
    iteration. In other words, there is a cyclic dependency between
    the remaining nodes that will remain unaltered, thus eliminating
    the possibility of reaching a fixpoint.
  \item (\emph{Only If}) Due to Theorem~\ref{stability},
    we know that we need at most $|V|$ computation steps, in order to
    reach a fixpoint, if at each computation step there exists at
    least a new node that gets stabilized. In other words, we need a
    finite number of steps to reach a fixpoint, if each step results
    in some progress. Thus, failure to reach a fixpoint requires an
    iteration where no progress has been made, i.e., no new nodes get
    stabilized.
  \end{enumerate}
\end{proof}

\noindent
Therefore, the optimization in Algorithm~\ref{alg:multiple:strat} of
detecting this exact condition (line~\ref{lst:line:unsat}) and
terminating would be triggered if and only if no fixpoint would be
reached whatsoever, had the algorithm continued its execution.

\begin{proof}[Soundness]
  We need to show that, if our algorithm computes a solution, this
  solution will be sound. Firstly, our algorithm maintains the
  invariant that $\forall (s,t) \in E$, node $s$ will
  eventually---i.e., once we reach a fixpoint---be placed somewhere lower
  than node $t$ (otherwise this condition would trigger yet another
  iteration). Therefore, our solution will contain no cycles since all
  of its edges will be facing \emph{upwards}, i.e., from a lower to a
  higher stratum. Furthermore, it is evident that, for each
  \emph{kp}-edge $(s,t)$, there will always exist a node $p \in
  \textit{proj}(s)$, such that $p$ will be placed at a lower stratum than $t$
  in our final solution. Thus, we can add the edge $(p,t)$ without
  introducing any cycles if none existed so far. This process will
  therefore generate a valid solution.
\end{proof}

\begin{proof}[Completeness]
  We need to show that, if a solution exists for a given constraint
  graph, then our algorithm will also be able to compute a solution,
  or equivalently (according to Theorem~\ref{termination}) that the
  stratification sequence being computed will reach a fixpoint.
  Consider such a (posited but unknown) solution. For such a solution
  we may generate a stratification (since the solution may contain no
  cycles), such that each of its edges is facing upwards and no empty
  strata exist, that is, $\forall (s,t) \in E_{sol} : S_{sol}[s] <
  S_{sol}[t]$, where $E_{sol}$ are the edges that form the solution,
  and $S_{sol}$ is its stratification.  We first show an important
  lemma.
%  Firstly, we will show that,
%  $\forall i \in \mathbb{N}, \forall v \in V : S_{i}[v] \leq
%  S_{sol}[v]$ for any such possible solution.


  \begin{lem}\label{lemma:minheight}
    Let $S_{sol}$ denote the stratification of a valid solution of the
    problem instance. We have that: $\forall i \in \mathbb{N}, \forall
    v \in V : S_{i}[v] \leq S_{sol}[v]$.
  \end{lem}

  \begin{proof}
    Suppose that there is a step $k \in \mathbb{N}$, such that
    it contains at least one node $u \in V$ with $S_k[u] > S_{sol}[u]$,
    and without loss of generality, suppose that $k$ is the smallest
    such integer, i.e., that before that point our stratification
    was upper bounded by that of the unknown solution:
    % I don't see this.
    % (which is safe to assume because of  Lemma~\ref{monotonicity}),
    $\forall j \in \{n \in \mathbb{N} \;|\; 0 \leq n < k \}, \forall v
    \in V : S_j[v] \leq S_{sol}[v]$. For $u$ to be placed at a higher
    stratum by our algorithm there must exist an edge $(s,u) \in E $
    such that either (i) $(s,u)$ was a constraining ordinary
    path-edge: $S_k[u] = S_{k-1}[s] + 1$, or (ii) $(s,u)$ was a
    \emph{kp}-edge and $\forall p \in \textit{proj}(s) : S_k[u] =
    S_{k-1}[p] + 1$. In the first case, we have that: $S_{sol}[u] \leq
    S_{k-1}[s] \leq S_{sol}[s]$, and since $(s,u)$ must also be
    present in the solution, this violates the single-direction edge
    property. In the second case, $S_{sol}[u] \leq S_{k-1}[p] \leq
    S_{sol}[p], \forall p \in \textit{proj}(s)$, which leads to
    another contradiction, since there must exist a node $p \in
    \textit{proj}(s)$, such that a path exists from $p$ to $u$ in the
    solution, which can only happen if $p$ was placed at a strictly
    lower stratum than $u$. Thus, since all possible cases lead to a
    contradiction, we have proved our initial proposition: our
    algorithm always assigns to every node a stratum that is lower
    than, or equal to, that of any true solution of the problem
    instance.
  \end{proof}

%% Single-use def
%We will use $S_{fix}$ to denote the final
%  stratification computed by our algorithm (when reaching a fixpoint).

%% The rest is trivial. Could be replaced by 1-2 sentences.
  \noindent
  Let $s_{sol} = \sum_{v \in V} S_{sol}[v]$ and $s_{i} = \sum_{v \in
    V} S_{i}[v], \forall i \in \mathbb{N}$. It follows that $s_i \leq
  s_{sol}, \forall i \in \mathbb{N}$, for any such possible
  solution. Additionally, because of Lemma~\ref{monotonicity} and our
  two theorems, we know
  that each step but the last will strictly increase the sum of all
  strata values over all nodes. E.g., if our algorithm ended its
  execution at step $n$,
%i.e., $S_{fix} = S_n$,
  then we would have: $s_0 < s_1 < \ldots < s_{n-1} = s_n$.

%\vfill\eject

  Suppose there is a solution but our algorithm fails to compute one
  (i.e., no fixpoint will ever be reached). Since $s_{i}$ strictly
  increases at each step of our algorithm, and the only way to return
  is by finding a valid solution, we know that there will exist a step
  $n$, such that $s_n > s_{sol}$. However, this contradicts our proven
  proposition that $s_i \leq s_{sol}, \forall i \in
  \mathbb{N}$. Therefore, we conclude that if valid a solution exists,
  our algorithm will also be able to compute one.

\end{proof}

\vfill\eject
\begin{thm}[Principality]
  Any solution produced by Algorithm~\ref{alg:multiple:strat} will
  have a minimum number of strata. That is, for any possible solution
  of the problem instance, with $t$ denoting the solution's total
  strata, we have that $n \leq t$, where $n$ is the total number of
  strata produced by Algorithm~\ref{alg:multiple:strat}.
\end{thm}

\begin{proof}
  Let $S_{sol}$ denote the stratification of a valid solution of the
  problem instance, and $t$ its total number of strata. Without loss
  of generality we assume that strata are denoted as consecutive
  integers starting from $0$, beginning from the lowest stratum. Thus,
  $\forall v \in V : S_{sol}[v] < t$.

  Since a solution exists and \emph{completeness} has been proved, we
  know that Algorithm~\ref{alg:multiple:strat} will also be able to
  terminate successfully at some step $n \in \mathbb{N}$, yielding its
  own solution. Let $n_{s}$ be the total number of strata, and $x \in
  V$ be a node at the highest stratum of the solution computed by our
  algorithm. That is, $\forall v \in V : S_{n}[v] \leq S_{n}[x]$. Node
  $x$ will also be the node that was changed last, which is at
  step $n-1$, i.e., $S_n[x] = S_{n-1}[x] \neq S_{n-2}[x]$. Therefore,
  from Corollary~\ref{maxstrata}, we have: $S_n[x] = S_{n-1}[x] = n -
  1$. Since strata are consecutive integers starting from $0$, we have
  that: $n_{s} = S_n[x] + 1 = n$.

  According to Lemma~\ref{lemma:minheight}, we have: $n = n_{s} =
  S_n[x] + 1 \leq S_{sol}[x] + 1 \leq t < t + 1$. Thus, we have shown
  that our algorithm minimizes the total number of strata it produces.
%  (a number common to all of its solutions).
\end{proof}
