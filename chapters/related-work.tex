\section{C/C++ Pointer Analysis}

We discussed some closely related work throughout the paper. Most C
and C++ analyses in the past have focused on scalability, at the
expense of precision. Several (e.g.,
\cite{antgrasshopper,popl/ZhengR08,popl/Lhotak11}) do not model more
than a small fraction of the functionality of modern intermediate
languages.

One important addition is the DSA work of Lattner et
al.~\cite{pldi/LattnerLA07}, which was the original points-to analysis
in LLVM. The analysis is no longer maintained, so comparing
experimentally is not possible. In terms of a qualitative comparison,
the DSA analysis is a sophisticated but ad hoc mix of techniques, some
of which add precision, while others sacrifice it for scalability. For
instance, the analysis is field-sensitive using byte offsets, at both
the source and the target of points-to edges. However, when a single
abstract object is found to be used with two different types, the
analysis reverts to collapsing all its fields. (Our analysis would
instead create two abstract objects for the two different types.)
Furthermore, the DSA analysis is unification-based (a Steensgaard
analysis), keeping coarser abstract object sets and points-to sets
than our inclusion-based analysis. Finally, the DSA analysis uses deep
context-sensitivity, yet discards it inside a strongly connected
component of methods.
%Overall, the DSA
%analysis is intriguing and it would be interesting to compare its
%precision with our approach, 

The field-sensitive inclusion-based analysis of Avots et
al. \cite{icse/AvotsDLL05} uses type information to improve its
precision. As in this work, they explicitly track the types of objects
and their fields, and filter out field accesses whose base object has
an incompatible type (which may arise due to analysis
imprecision). However, their approach is array-insensitive and does
not employ any kind of type back-propagation to create more
(fine-grained) abstract objects for polymorphic allocation
sites. Instead, they consider objects used with multiple types as
possible type violations. Finally, they extend type compatibility with
a form of structural equivalence to mark types with identical physical
layouts as compatible. The implementation of \cclyzer{} applies a more
general form of type compatibility, presented in
Section~\ref{structsens/sect/addons}.

\citeauthor{lctrts/Mine06} \cite{lctrts/Mine06} presents a highly
precise analysis, expressed in the abstract interpretation framework,
that translates any field and array accesses to pointer arithmetic. By
relying on an external numerical interval analysis, this technique is
able to handle arbitrary integer computations, and, thus, any kind of
pointer arithmetic. However, the precision comes with scalability and
applicability limitations: the technique can only analyze programs
without dynamic memory allocation or recursion.
%(which commonly applies only to embedded critical software).

There are similarly other C/C++-based analyses that claim
field-sensitivity~\cite{popl/HardekopfL09,cgo/HardekopfL11}, but it is
unclear at what granularity this is implemented. Existing descriptions
in the literature do not match the precision of our
structure-sensitive approach, which maintains maximal structure
information (with typed abstract objects and full distinction of subobjects), at
both sources and targets of points-to relationships. Nystrom et
al.~\cite{paste/NystromKH04} have a fine-grained heap abstraction that
corresponds to standard use of ``heap cloning''
(a.k.a. ``context-sensitive heap'').% techniques.



\section{Miscellaneous}

%% -----------------------------------------------------------------------------
%   CFL Reachability Problems
%% -----------------------------------------------------------------------------

\paragraph{CFL reachability formulation.} In
Chapter~\ref{chapter:structsens}, we formulated pointer analysis
algorithms in Datalog. Employing a restricted language not only offers
guarantees of termination and complexity bounds, but also permits more
aggressive optimization of the language features.

Along these lines, pointer analysis and other related analyses have
been formulated as a \emph{context-free language (CFL) reachability}
problem. The idea is that we may encode an input program as a labeled
graph, and a specific analysis corresponds to the definition of a
context-free grammar, \(G\). The relation being computed holds for two
nodes of the graph if and only if there exists a path from one node to
the other, such that the concatenation of the labels of edges along
the path belongs in the language \(L(G)\) defined by \(G\).

Specifically, the input graph normally consists of nodes representing
program elements such as variables, types, methods, statements, and so
on. Edges represent relations between those program elements. For
instance, an edge \((s,\,t)\) may represent that there exists an
assignment from variable \(s\) to variable \(t\). Moreover, edges may
encode field accesses (load/store), method invocations, pointer
dereferences, etc, and hence may even connect different \emph{kinds}
of program elements. The exact choice of domains depends on the
specific analysis being run and the problem it addresses. Since we
want to express many input relations, we need many types of edges,
represented as labels (e.g., we can label a field access edge by some
symbol denoting field access plus the name of the field).  For a given
analysis, a context-free grammar \(G\) encodes the desired computed
relations (e.g., which pointers are memory aliases) as non-terminal
symbols, and supplies production rules that express how they relate to
the simpler relations represented by graph edges (terminals). The CFL
reachability answer is then commonly computed by employing a dynamic
programming algorithm.

The first application of such a framework in program analysis was
designed to solve various interprocedural dataflow-analysis problems
\cite{popl/RepsHS95}, but CFL reachability has since been used in a
wide range of problems, such as:
\begin{inparaenum}[(i)]
\item the computation of points-to relations
  \cite{journals/infsof/Reps98},
\item the (demand-driven) computation of may-alias pairs for a C-like
  language \cite{popl/ZhengR08},
\item Andersen-style pointer analysis for Java
  \cite{oopsla/SridharanGSB05}.
\end{inparaenum}

Any CFL reachability problem can be converted to a Datalog program
\cite{journals/infsof/Reps98}, but the inverse is not true (i.e., CFL
reachability corresponds to a restricted class of Datalog programs,
the so-called ``chain programs''). Thus, the primary advantage of CFL
reachability is that it permits more efficient implementations.

A \emph{chain program} consists of rules of the form:

\[p(X, Y) \leftarrow q_0(X,Z_1),\, q_1(Z_1, Z_2),\, \dots,\, q_k(Z_k,
  Y). \]

We can express a CFL reachability problem in Datalog by using such a
chain rule to represent the following production of grammar \(G\):

\[p \rightarrow q_0\; q_1\; \dots\; q_k \]

where a Datalog fact \(e(m, n)\) represents an edge \((m, n)\) labeled
\(e\) in the graph.

An even more restrictive variant, \emph{Dyck-CFL reachability}, can be
obtained by restricting the underlying CFL to a Dyck language, i.e.,
one that generates balanced-parentheses expressions. Although
restrictive, this approach still suffices for some simple pointer
analysis algorithms, while allowing very aggressive optimization,
often with impressive performance gains \cite{pldi/ZhangLYS13}.

% -CFL formulations (Dyck-CFL PLDI'13, Rugina, Bodik)
% \citep{pldi/ZhangLYS13}             % fast Dyck-CFL
% \citep{popl/ZhengR08}               % demand-driven; may-alias pairs
% \citep{journals/infsof/Reps98}      % also published in SLP' 1997; points-to relations
% \citep{popl/RepsHS95}               % interprocedural dataflow-analysis problems
% \citep{cc/Reps94}                   % magic-sets transformation -> demand-driven
% \citep{oopsla/SridharanGSB05}       % Java; demand-driven; balanced parentheses (field accesses)



%% -----------------------------------------------------------------------------
%   Shape Analysis
%% -----------------------------------------------------------------------------

\paragraph{Shape Analysis.}

So far, the techniques we have presented for pointer analysis have
been based on the allocation sites as the primary source for inventing
new abstract objects in memory. Despite our deviation from the
standard allocation-site abstraction, where a single abstract object
will be allocated per allocation site, even our own techniques
described in Chapter~\ref{chapter:structsens} will use the allocation
site as the basis of abstraction (but may create multiple objects per
site, nonetheless). A different approach altogether is that of the
techniques termed as \emph{shape analysis}
\cite{toplas/SagivRW02,popl/SagivRW99,sas/ManevichSRF04,sas/Lev-AmiS00,toplas/SagivRW98,sefm/FerraraFJ12}.
The primary goal of standard shape analysis is to be able to infer the
\emph{shapes} of objects in memory. E.g., to be able to detect if some
objects form a list or a tree, if some list may contain cycles, if a
subtree or a portion of a list may be shared (i.e., be reachable from
multiple objects), and so on.\footnote{The term shape analysis is
  quite generic (i.e., any analysis designed to infer the shapes of
  objects) and has been examined in many different contexts
  \cite{popl/GhiyaH96}. Here, we focus on \emph{parametric shape
    analysis via 3-valued logic}, as one of the most notable
  methodologies in that area.}

To achieve such a feat, shape analysis associates a list of properties
to each abstract memory object (e.g., \emph{pointed by variable
  \var{v}}, \emph{transitively reachable by variable \var{r}}, and so
on) and uses Kleene's three-valued logic to differentiate between must
and may information. For instance, if the ``\emph{pointed by variable}
\(\var{v}\)'' property of an abstract object \(\alloc{obj}\) has the
value 1 at some memory state, it means that variable \var{v}
\emph{must} point to this object (at the given state), whereas the
value \(1\over{2}\) would represent our familiar \emph{may} notion of
points-to. Hence, shape analysis performs an amalgam of must and may
analysis simultaneously.

At each memory state the analysis has computed, it tries to collapse
\(1\over{2}\) values of properties to either 0 or 1, via the so called
\emph{focus operation}. Inconsistent states are then discarded at the
\emph{coerce operation}. Thus, the analysis dynamically tries to
eliminate uncertainty by \emph{focusing} on the values of some core
predicates (and statement-specific formulas), at the expense of
possible memory state explosion---the abstract interpretation of each
program statement tends to create multiple output states for each one
of its input states.  As for abstract objects, they are defined only
by the values they have for some basic properties (called
\emph{abstraction properties}) of the particular analysis. Therefore,
the upper bound for the number of abstract objects in memory is
exponential with respect to the number of abstraction predicates
defined. The latter will almost certainly include a predicate per each
program variable.

By applying these techniques, and choosing the right abstraction
predicates, the analysis will be able to carve out the \emph{shapes}
of objects in memory. For instance, given an input memory state where
variable \var{v} points to the head of some list \(l\) with at least
two elements, the abstract interpretation of instruction ``\code{v =
  v->next}'' will create multiple output memory states where either:
\begin{inparaenum}[(i)]
\item \var{v} points to a new head of a list with a non-empty tail
  (corresponding to the case where \(l\) contained at least three
  elements), or
\item \var{v} points to the single element of a list with no tail at
  all (corresponding to the case where \(l\) contained exactly two
  elements).
\end{inparaenum}
In both output memory states, the analysis will compute values of 1
for the predicate ``\emph{pointed by variable \var{v}}''; hence, it
will know exactly where \var{v} points to and not lose any precision,
at the expense of increasing the possible memory states by 1.

\citeauthor{popl/SagivRW99} initiated the field of parametric shape
analysis via 3-valued logic
\cite{toplas/SagivRW98,popl/SagivRW99,toplas/SagivRW02}, and
\citeauthor{sas/Lev-AmiS00} presented the TVLA framework for shape
analysis \cite{sas/Lev-AmiS00}. Since then, there has been work on
various extensions such as
\begin{inparablank}
\item a more economic heap abstraction \cite{sas/ManevichSRF04},
\item better support for recursive programs \cite{cc/RinetzkyS01}
\item and programs with highly nested data structures
  \cite{cav/BerdineCCDOWY07} , and
\item the incorporation of a value analysis into the shape analysis
  algorithm~\cite{sefm/FerraraFJ12},
\end{inparablank}
among others.


%% -----------------------------------------------------------------------------
%   Separation Logic
%% -----------------------------------------------------------------------------

\paragraph{Separation Logic.}
Points-to analysis provides a model of the heap (or of memory in
general, for a language such as C). Other approaches for heap analysis
that can be used to prove pointer safety are based on the field of
\emph{separation logic}. Separation logic, in turn, can be viewed as
an extension of Hoare logic
\cite{journals/cacm/Hoare69,floyd1967assigning,lics:2002/Reynolds,csl/OHearnRY01}.
Hoare logic is a formal system for reasoning about the correctness of
programs, by encoding the programming language's semantics in
\emph{Hoare tiples}. A Hoare triple has the form \(\{P\}\, C\, \{Q\}\)
and describes how the execution of a command changes the state of the
computation. Specifically, it states that whenever the assertion \(P\)
holds, before executing command \(C\), then assertion \(Q\) will hold
afterwards (if \(C\) terminates). The \(P,\, Q\) assertions can
express conditions on program variables, written by using standard
mathematical notations together with logical operators (or, in
general, some form of calculus like \emph{first-order logic}).

Hoare logic provides two ways to generate verification conditions:
\begin{inparaenum}[(i)]
\item either forwards, by starting from a precondition and
  generating formulas to prove a postcondition
\item or backwards, by starting from a postcondition and trying to
  prove a precondition.
\end{inparaenum}
Either way, in the general case, it cannot provide fully automated
reasoning; building a proof may require human guidance.

% Even though there has been effort in increasing the level of
% automation (e.g., for the (semi-)automatic inference of loop
% invariants)

Separation logic
\cite{lics:2002/Reynolds,csl/OHearnRY01,popl/IshtiaqO01,Reynolds00intuitionisticreasoning}
extends Hoare logic by introducing additional operators in the syntax
of assertions, that facilitate local reasoning. Namely, the
\emph{separating conjunction} \(\,P * \,Q\) asserts that \(P\) and
\(Q\) hold for separate portions of memory, and thus can be used on
program-proof rules to provide modular reasoning about
programs. Additional operators include the separating implication,
\(P \sep Q\), which asserts that if the current heap is extended with
a disjoint part in which \(P\) holds, then \(Q\) will hold in the
extended heap, and more. Note that, these operators do not increase
the ``completeness'' of Hoare logic---what can be proven in separation
logic can also be proven in Hoare logic. Rather, they merely simplify
the specifications and proofs.

% citation needed
% Peter O'Hearn
% A Primer on Separation Logic (and Automatic Program Verification and Analysis)

\citeauthor{popl/CalcagnoDOY09} present a compositional shape analysis
\cite{popl/CalcagnoDOY09} to be used in (lightweight) program
verification that builds on these concepts of separation logic
(instead of the TVLA approach of \citeauthor{popl/SagivRW99}). In
classical logic, \emph{abduction} stands for the inference of
``missing'' assumptions \(M\), such that, given another assumption
\(A\) and a goal \(G\), one can prove \(G\) by synthesizing \(A\) and
\(M\):
\[A \,\land M\, \;\vdash\; G \]

A similar problem can be phrased by using separating conjunction
instead of classical conjunction, which also partitions the premises:
\[A \;*\; ??\, \;\vdash\; G \]

This finally leads to the more general problem of \emph{bi-abduction},
if we allow for leftover portions of state (the frame):
\[A \,*\, \textit{?anti-frame} \;\vdash\; G \,*\, \textit{?frame} \]

The notion of bi-abduction is used as the basis of a new compositional
interprocedural shape analysis algorithm.



%%% Local Variables:
%%% mode: latex
%%% TeX-master: "../thesis"
%%% End:
