\paragraph{CFL reachability formulation.} In
Chapter~\ref{chapter:structsens}, we formulated pointer analysis
algorithms in Datalog. Employing a restricted language not only offers
guarantees of termination and complexity bounds, but also permits more
aggressive optimization of the language features.

Along these lines, pointer analysis and other related analyses have
been formulated as a \emph{context-free language (CFL) reachability}
problem. The idea is that we may encode an input program as a labeled
graph, and a specific analysis corresponds to the definition of a
context-free grammar, \(G\). The relation being computed holds for two
nodes of the graph if and only if there exists a path from one node to
the other, such that the concatenation of the labels of edges along
the path belongs in the language \(L(G)\) defined by \(G\).

Specifically, the input graph normally consists of nodes representing
program elements such as variables, types, methods, statements, and so
on. Edges represent relations between those program elements. For
instance, an edge \((s,\,t)\) may represent that there exists an
assignment from variable \(s\) to variable \(t\). Moreover, edges may
encode field accesses (load/store), method invocations, pointer
dereferences, etc, and hence may even connect different \emph{kinds}
of program elements. The exact choice of domains depends on the
specific analysis being run and the problem it addresses. Since we
want to express many input relations, we need many types of edges,
represented as labels (e.g., we can label a field access edge by some
symbol denoting field access plus the name of the field).  For a given
analysis, a context-free grammar \(G\) encodes the desired computed
relations (e.g., which pointers are memory aliases) as non-terminal
symbols, and supplies production rules that express how they relate to
the simpler relations represented by graph edges (terminals). The CFL
reachability answer is then commonly computed by employing a dynamic
programming algorithm.

The first application of such a framework in program analysis was
designed to solve various interprocedural dataflow-analysis problems
\cite{popl/RepsHS95}, but CFL reachability has since been used in a
wide range of problems, such as:
\begin{inparaenum}[(i)]
\item the computation of points-to relations
  \cite{journals/infsof/Reps98},
\item the (demand-driven) computation of may-alias pairs for a C-like
  language \cite{popl/ZhengR08},
\item Andersen-style pointer analysis for Java
  \cite{oopsla/SridharanGSB05}.
\end{inparaenum}

Any CFL reachability problem can be converted to a Datalog program
\cite{journals/infsof/Reps98}, but the inverse is not true (i.e., CFL
reachability corresponds to a restricted class of Datalog programs,
the so-called ``chain programs''). Thus, the primary advantage of CFL
reachability is that it permits more efficient implementations.

A \emph{chain program} consists of rules of the form:

\[p(X, Y) \leftarrow q_0(X,Z_1),\, q_1(Z_1, Z_2),\, \dots,\, q_k(Z_k,
  Y). \]

We can express a CFL reachability problem in Datalog by using such a
chain rule to represent the following production of grammar \(G\):

\[p \rightarrow q_0\; q_1\; \dots\; q_k \]

where a Datalog fact \(e(m, n)\) represents an edge \((m, n)\) labeled
\(e\) in the graph.

An even more restrictive variant, \emph{Dyck-CFL reachability}, can be
obtained by restricting the underlying CFL to a Dyck language, i.e.,
one that generates balanced-parentheses expressions. Although
restrictive, this approach still suffices for some simple pointer
analysis algorithms, while allowing very aggressive optimization,
often with impressive performance gains \cite{pldi/ZhangLYS13}.

% -CFL formulations (Dyck-CFL PLDI'13, Rugina, Bodik)
% \citep{pldi/ZhangLYS13}             % fast Dyck-CFL
% \citep{popl/ZhengR08}               % demand-driven; may-alias pairs
% \citep{journals/infsof/Reps98}      % also published in SLP' 1997; points-to relations
% \citep{popl/RepsHS95}               % interprocedural dataflow-analysis problems
% \citep{cc/Reps94}                   % magic-sets transformation -> demand-driven
% \citep{oopsla/SridharanGSB05}       % Java; demand-driven; balanced parentheses (field accesses)


%%% Local Variables:
%%% mode: latex
%%% TeX-master: "../thesis"
%%% End:
