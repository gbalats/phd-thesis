\documentclass{llncs}

\usepackage{amssymb,amsmath,latexsym}
\usepackage[chapter]{algorithm}
\usepackage{algorithmicx}
\usepackage{algpseudocode}
\usepackage{alltt}
\usepackage{bookmark}
\usepackage{booktabs}
\usepackage{color,soul}
\usepackage{comment}
\usepackage{fancyhdr}
\usepackage{footnote}
\usepackage{graphicx}
\usepackage{marvosym}
\usepackage{multirow}
\usepackage{multicol}
\usepackage{mathtools}
\usepackage{epigraph}
\usepackage[defblank]{paralist}
% \usepackage{boxedminipage}
% \usepackage{varwidth}
\usepackage{url}
\usepackage{subcaption}
% \usepackage{cite}
% \captionsetup{compatibility=false}
\usepackage{verbatimbox}
\usepackage{minted}
\usemintedstyle{friendly}             % set minted colour theme
\usepackage{xparse}
\usepackage{hyperref}
\hypersetup{
    unicode=true,                     % non-Latin characters in bookmarks
    pdffitwindow=true,                % page fit to window when opened
    pdfnewwindow=true,                % links in new window
    pdfkeywords={},                   % list of keywords
    colorlinks=true,                  % false: boxed links; true: colored links
    linkcolor=black,                  % color of internal links
    citecolor=black,                  % color of links to bibliography
    filecolor=black,                  % color of file links
    urlcolor=black,                   % color of external links
    pdftitle={},                      % title
    pdfauthor={},                     % author
    pdfsubject={}                     % subject of the document
}
\usepackage{csquotes}

% \usepackage{datetime}
% \usepackage{booktabs}
% \usepackage{float}
% \usepackage{upquote}
% \usepackage{color}
% \usepackage{dcolumn}
% \usepackage{textcomp}
% \usepackage{dcolumn}
% \usepackage{chngcntr}
% \usepackage{xspace}
% \usepackage{tikz}
% \usepackage{rotating}
% \usepackage{array}
% \usepackage{microtype}
% \usepackage{float}

\newcommand{\doop}{\textsc{Doop}}

\newcommand{\cclyzer}{\texttt{cclyzer}}
\newcommand{\pearce}{\emph{Pearce$^c$}}
\newcommand{\code}[1]{\texttt{#1}}
\newcommand{\pt}[1]{\texttt{\textsc{Pt}(#1)}}
\newcommand{\javaclass}[1]{\url{#1}}
\newcommand{\javamethod}[1]{\url{#1}}
\newcommand{\javasignature}[1]{\url{#1}}

% \usepackage{../styles/instructions}
\usepackage{../styles/languages}
% \usepackage{../styles/mathmisc}
% \usepackage{../styles/algorithms}


\begin{document}
\pagestyle{headings}

\mainmatter 

\title{Recovering Structural Information for Better Static Analysis}
\author{George Balatsouras\thanks{Dissertation Advisor: Yannis Smaragdakis, Professor}}

\institute{National and Kapodistrian University of Athens \\
Department of Informatics and Telecommunications \\
\email{gbalats@di.uoa.gr}}

\maketitle              

\pagestyle{empty}

\begin{abstract}

  To reach a truly broad level of program understanding, static
  analysis techniques need to create an abstraction of memory that
  covers all possible executions. Such abstract models may quickly
  degenerate after losing essential structural information about the
  memory objects they describe, due to the use of specific programming
  idioms and language features, or because of practical analysis
  limitations. In many cases, some of the lost memory structure may be
  retrieved, though it requires complex inference that takes advantage
  of indirect uses of types. Such recovered structural information
  may, then, greatly benefit static analysis.
  %
  This dissertation shows how we can recover structural information,
  first
  \begin{inparaenum}[(i)]
  \item in the context of C/C++, and next, in the context of
    higher-level languages without direct memory access, like Java,
    where we identify two primary causes of losing memory structure:
  \item the use of reflection, and
  \item analysis of partial programs.
  \end{inparaenum}
  We show that, in all cases, the recovered structural information
  greatly benefits static analysis on the program.

  \keywords{Pointer Analysis; Object-Oriented Programming; Type %
    Hierarchy; Reflection}

\end{abstract}

\section{Introduction}

The most promising and powerful of existing static analysis techniques
rely on the creation of some \emph{abstract memory model} of the
program. What objects will the memory contain, at some state of
execution? What will their structure be like?  A faithful abstract
representation of the actual memory is, however, a demanding task; its
precision often decisive for the value of whatever the static
analysis is aiming to eventually compute (be it the identification of
complex bug patterns or the opportunities for effective
optimizations).

\paragraph*{Thesis.}
\begin{displayquote}
  There is \emph{implicit structural information} in the program,
  about the memory it will allocate, that can improve the quality of
  the abstract memory model constructed by static analysis. This
  structural information is not readily available, but may be
  recovered via inference, primarily by tracking the use of types in
  the program.
\end{displayquote}

\noindent
We provide a number of techniques that recover such
lost memory structure, in two different settings:
\begin{inparaenum}[(1)]
\item in C/C++ programs, as a typical case of low-level code with
  direct memory access, where the program's memory structure is often
  lost due to specific programming idioms and the inherent low-level
  nature of the language, and
\item in Java programs, where, despite the high-level nature of the
  language, structural information may be lost
  \begin{inparaenum}[(a)]
  \item for \emph{partial programs} (i.e., libraries or any programs that
    lack some of their parts), which, in the form of Java Archives
    (JARs), constitute the main distributable code entity of this
    managed language, or
  \item due to Java's \emph{reflection} mechanism, which allows
    runtime inspection of classes, interfaces, fields and methods, and
    can be used to instantiate new objects, invoke methods, get/set
    field values, and so on, without exact static type information
    (e.g., the name of the method to be invoked can be created
    dynamically using plain string operations).
  \end{inparaenum}
\end{inparaenum}

\section{Structure-Sensitive Points-To Analysis for C and C++}

Points-to analysis computes an abstract model of the memory that is
used to answer the following query: \emph{What can a pointer variable
  point-to, i.e., what can its value be when dereferenced during
  program execution?}  This query serves as the cornerstone of
many other static analyses aiming to enhance program understanding or
assist in bug discovery (e.g., deadlock detection), by computing
higher-level relations that derive from the computed points-to
sets. In the literature, one can find a multitude of points-to
analyses with varying degrees of precision and speed.

One of the most popular families of pointer analysis algorithms,
\emph{inclusion-based} analyses (or Andersen-style analyses
\cite{andersen:thesis}), originally targeted the C language, but has
been extended over time and successfully applied to higher-level
object-oriented languages, such as Java
\cite{pldi/BerndlLQHU03,oopsla/BravenboerS09,issta/MilanovaRR02,oopsla/RountevMR01,oopsla/WhaleyR99}.
Surprisingly, precision-enhancing features that are common practice in
the analysis of Java programs, such as field sensitivity or online
call-graph construction are absent in many analyses of C/C++
\cite{antgrasshopper,toplas/HindBCC99,popl/ZhengR08,pldi/HeintzeT01a,pldi/Das00,sas/HardekopfL07}.

In the case of field sensitivity, the reason behind its frequent
omission when analyzing C is that it is much harder to implement
correctly than in Java. As noted by Pearce et
al. \cite{toplas/PearceKH07}, the crucial difference is that, in C/C++,
it is possible to have the address of a field taken, stored to some
pointer, and then dereferenced later, at an arbitrarily distant
program point. In contrast, Java does not permit taking the address of
a field; one can only load or store to some field directly. Hence,
\code{load/store} instructions in Java bytecode (or any equivalent IR)
need an extra field specifier, whereas in C/C++ intermediate
representations (e.g., LLVM bitcode) \code{load/store} requires only a
single address operand. The precise field affected is not explicit, but
only possibly computed by the analysis itself.

The effect of such difference in the underlying IRs, as far as pointer
analysis is concerned, is far from trivial. In C, the computed
points-to sets have an expanded domain, since now the analysis must
be able to express that a variable \code{p} \emph{at some offset}
\code{i} may point-to another variable \code{q} \emph{at some offset}
\code{j}, with these offsets corresponding to either field components or
array elements.

The best-documented approach on how to incorporate
field sensitivity in a C/C++ points-to analysis is that of Pearce et
al. \cite{paste/PearceKH04,toplas/PearceKH07}. The authors extend the
constraint-graph of the analysis by adding (positive) weights to 
edges; the weights correspond to the respective field indices. For
instance, the instruction ``\code{q = \&(p->f$_{\code{i}}$)}'' would
be encoded as a constraint $q \supseteq p + i$. However, this approach
does not take types into account. In fact, types are not even
statically available at all allocation sites, since most standard C
allocation routines are type-agnostic and return byte arrays that are
cast to the correct type at a later point (e.g., \code{malloc()},
\code{realloc()}, \code{calloc()}).  Thus, field $i$ is represented
with no regard to the type of its base object, even when this base
object abstracts a number of concrete objects of different types.
The lack of type information for abstract objects is a great source of
imprecision, since it results in a prohibitive number of spurious
points-to inferences.

We argue that type information is an essential part in increasing
analysis precision, even when it is not readily available. The
abstract object types should be rigorously recorded in all cases,
especially when indexing fields, and used to filter the points-to
sets. In this spirit, we present a \emph{structure-sensitive} analysis
for C/C++ that employs a number of techniques in this direction,
aiming to retrieve high-level structure information for abstract
objects in order to increase analysis precision:

\begin{enumerate}
\setlength\itemsep{0.5em}
\item First, the analysis records the type of an abstract object when this type is
  available at the allocation site. This is the case with stack
  allocations, global variables, and calls to C++'s \code{new()}
  heap allocation routine.
\item In cases where the type is not available (as in a call to
  \code{malloc()}), the analysis deviates from the allocation-site
  abstraction and creates multiple abstract objects per allocation
  site: one for every type that the object could have. Thus, each
  abstract object of type \code{T} now represents the set of all
  concrete objects of type \code{T} allocated at this site. To
  determine the possible types for a given allocation site, the
  analysis creates a special type-less object and records the cast
  instructions it flows to (i.e., the types it is cast to), using the
  existing points-to analysis. This is similar to the use-based
  \emph{back-propagation} technique used in past work
  \cite{ecoop/LiTSX14,aplas/LivshitsWL05,aplas/SmaragdakisBKB15}, in a
  completely different context---handling Java reflection.
\item The field components of abstract objects are represented as
  abstract objects themselves, as long as their type can be
  determined. That is, an abstract object \code{SO} of struct type
  \code{S} will trigger the creation of abstract object
  \code{SO.f$_\code{i}$}, for each field \code{f$_\code{i}$} in
  \code{S}. (The aforementioned special objects trigger no such field
  component creation, since they are typeless.)
%  we cannot determine either their type or
%  their resulting fields' types. 
  Thus, the recursive creation of subobjects is bounded by the type
  system, which does not allow the declaration of types of infinite
  size.
\item Finally, the analysis treats array elements similarly to
  field components (i.e., by representing them as distinct abstract
  objects, if we can determine their type), as long as their respective
  indices statically appear in the source code. That is, an abstract
  object \code{AO} of array type \code{[T$\times$N]} will trigger the
  creation of abstract object \code{AO[c]}, if the constant \code{c}
  is used to index into type \code{[T$\times$N]}. The object
  \code{AO[*]} is also created, to account for indexing at unknown
  (variable) indices.
\end{enumerate}

\noindent
The last point offers some form of array-sensitivity
as well and is crucial for analyzing C++ code, lowered to an
intermediate representation such as LLVM bitcode, in which all the
object-oriented features have been translated away. To be able to
resolve virtual calls, an analysis must precisely reason about the
exact v-table index that a variable may point to, and the method that
such an index may itself point-to. That is, a precise analysis should
not merge the points-to sets of distinct indices of v-tables.

We offer an implementation of our approach over the full LLVM bitcode
intermediate language, in the form of a new static analysis tool,
\cclyzer{}\footnote{\cclyzer{} is publicly available at
  \url{https://github.com/plast-lab/cclyzer}}.  We show that our
approach yields much higher precision than past analyses, allowing
accurate distinctions between subobjects, v-table entries, array
components, and more. Especially for C++ programs, this precision is
invaluable for a realistic analysis. Compared to the state-of-the-art
past approach, our techniques exhibit substantially better precision
along multiple metrics and realistic benchmarks (e.g., 40+\% more
variables with a single points-to target).

\section{More Sound Static Handling of Java Reflection}

Moving to higher-level languages, like Java, we note that essential
structural information is often lost in Java programs too, yet for
different reasons. A source of analysis imprecision, especially in
determining the types of abstract objects constructed by the analysis,
lies in the use of Java's reflection mechanism: the ability to inspect
and dynamically retrieve classes, methods, attributes, etc. at
runtime.

By using the Reflection API, Java programs can encompass dynamic
behavior. However, statically reasoning about the behavior of software
that uses reflection can be especially cumbersome.  Unfortunately,
reflection is ubiquitous in large Java programs.
%
When a Java program accesses a class by supplying its name as a
run-time string, via the \javasignature{Class.forName} library call,
the static analysis has very few available courses of action: It needs
to either conservatively over-approximate (e.g., assume that
\emph{any} class can be accessed, possibly limiting the set later,
after the returned object is used), or to perform a string analysis
that will allow it to infer the contents of the \code{forName} string
argument. Both options can be detrimental to the scalability of the
analysis: the conservative over-approximation may never become
constrained enough by further instructions to be feasible in practice;
precise string analysis is impractical for programs of realistic size.
It is telling that \emph{no practical Java program analysis framework
  in existence handles reflection soundly} \cite{soundiness15},
although other language features are modeled soundly.\footnote{In our
  context, \emph{sound} = over-approximate, i.e., guaranteeing that
  all possible behaviors of reflection operations are modeled.}

%
Full soundness is not practically achievable, but it can still be
approximated for the well-behaved reflection patterns encountered in
regular, non-adversarial programs.  Therefore, it makes sense to treat
soundness as a continuous quantity: something to improve on, even
though we cannot perfectly reach.  To avoid confusion, we use the term
\emph{empirical soundness} for the quantification of how much of the
dynamic behavior the static analysis covers. Computable metrics of
empirical soundness can help quantify how close an analysis is to the
fully sound result. Based on such metrics, one can make comparisons
(e.g., ``more sound'') to describe soundness improvements.

The second challenge of handling reflection in a static analysis is
\emph{scalability}.  The online documentation of the IBM \textsc{Wala}
library~\cite{www:wala-reflection} concisely summarizes the current
state of the practice, for \emph{points-to analysis} in the Java
setting.

\begin{quote}
  \emph{Reflection usage and the size of modern libraries/frameworks
    make it very difficult to scale flow-insensitive points-to
    analysis to modern Java programs. For example, with default
    settings, \textsc{Wala}'s pointer analyses cannot handle any
    program linked against the Java 6 standard libraries, due to
    extensive reflection in the libraries.}
\end{quote}

\noindent The same caveats routinely appear in the research
literature. Multiple published points-to analysis papers analyze
well-known benchmarks with reflection
disabled~\cite{popl/SmaragdakisBL11,pldi/KastrinisS13,ecoop/AliL12,ecoop/AliL13}.


A representative quote~\cite{popl/SmaragdakisBL11} illustrates:
\begin{quote}
  \emph{Hsqldb and jython could not be analyzed with reflection
    analysis enabled [...]  ---hsqldb cannot even be analyzed
    context-insensitively and jython cannot even be analyzed with the
    1obj analysis. This is due to vast imprecision introduced when
    reflection methods are not filtered in any way by constant strings
    (for classes, fields, or methods) and the analysis infers a large
    number of reflection objects to flow to several variables.  [...]
    For these two applications, our analysis has reflection reasoning
    disabled.  Since hsqldb in the DaCapo benchmark code has its main
    functionality called via reflection, we had to configure its entry
    point manually.}
\end{quote}

\noindent
We describe an approach for handling reflection with improved empirical
soundness (as measured against prior approaches and dynamic
information), again, in the context of a points-to analysis. Our
approach is based on the combination of string-flow and points-to
analysis from past literature augmented with
\begin{inparaenum}[(a)]
\item substring analysis and modeling of partial string flow through
  string builder classes;
\item new techniques for analyzing reflective entities based on
  information available at their use-sites.
\end{inparaenum}
% The resulting analysis is general, without any need for hand-tuning. 
In experimental comparisons with prior approaches, we demonstrate a
combination of both improved soundness (recovering the majority of
missing call-graph edges) and increased performance.
%
Our approach requires no manual configuration and achieves
significantly higher empirical soundness without sacrificing
scalability, for realistic benchmarks and libraries (DaCapo Bach and
Java 7).

In experimental comparisons with the recent \textsc{Elf}
system~\cite{ecoop/LiTSX14} (itself improving over the reflection
analysis of the \textsc{Doop} framework~\cite{oopsla/BravenboerS09}),
our algorithm discovers most of the call-graph edges missing (relative
to a dynamic analysis) from \textsc{Elf}'s reflection analysis.  This
improvement in empirical soundness is accompanied by \emph{increased}
performance relative to \textsc{Elf}, demonstrating that near-sound
handling of reflection is often practically possible. Concretely, our
work for reflection:
\begin{itemize}[\(\cdot\)]
\item introduces key techniques in static reflection handling that
  contribute greatly to empirical soundness. The techniques generalize
  past work from an intra-procedural to an inter-procedural setting
  and combine it with a string analysis;
\item shows how scalability can be addressed with appropriate tuning
  of the above generalized techniques;
\item thoroughly quantifies the empirical soundness of a static
  points-to analysis, compared to past approaches and to a dynamic
  analysis;
\item is implemented and evaluated on top of an existing open
  framework (\textsc{Doop}~\cite{oopsla/BravenboerS09}).
  % This can offer a platform for experimentation with sophisticated
  % static reflection handling, possibly also leading to good static
  % handling of other dynamic features (e.g., complex dynamic
  % loading).
\end{itemize}

\section{Class Hierarchy Complementation for Java}

Whole-program static analysis is essential for clients that require
high-precision and a deeper understanding of program behavior. Modern
applications of program analysis, such as large scale refactoring
tools, race and deadlock detectors, and security vulnerability
detectors, are virtually inconceivable without whole-program analysis.

For whole-program analysis to become truly practical, however, it
needs to overcome several real-world challenges. One of the somewhat
surprising real-world observations is that whole-program analysis
requires the availability of much more than the ``whole program''.
The analysis needs an overapproximation of what constitutes the
program. Furthermore, this overapproximation is not merely
what the analysis computes to be the ``whole program'' after it
has completed executing. Instead, the overapproximation needs to be
as conservative as required by any intermediate step of the analysis,
which has not yet been able to tell, for instance, that some method
is never called.

Consider the example of trying to analyze a program $P$ that uses a
third-party library $L$. Program $P$ will likely only need small parts
of $L$.  However, other, entirely separate, parts of $L$ may make use
of a second library, $L'$.  It is typically not possible to analyze
$P$ with a whole program analysis framework without also supplying the
code not just for $L$ but also for $L'$, which is an unreasonable
burden. In modern languages and runtime systems, $L'$ is usually not
necessary in order to either compile $P$ or run it under any
input. The problem is exacerbated in the current era of large-scale
library reuse.  In fact, it is often the case that the user is not
even aware of the existence of $L'$ until trying to analyze $P$.

Our research consists precisely of addressing such need in full
generality. \emph{Given a set of Java class and interface definitions,
  in bytecode form, we compute a ``program complement'', i.e.,
  skeletal versions of any referenced missing classes and interfaces
  so that the combined result constitutes verifiable Java bytecode.}

To see why the problem has interesting depth and complexity,
consider a simple fragment of Java bytecode and the constraints it
induces. Our convention here is that single-letter class names at
the lower end of the alphabet (\code{A}, \code{B}, ...)  correspond
to known types, while class names at the high end of the alphabet
(\code{X}, \code{Y}, \code{Z}) denote phantom types.  We present
bytecode in a slightly condensed form, to make clear what method
names or type names are referenced in every instruction.

\begin{bytecode}
  public void foo(X, Y)
  0: aload_2     // load on stack 2nd argument (of type Y)
  1: aload_1     // load on stack 1st argument (of type X)
  2: invokevirtual X.bar:(LA;)LZ; // method 'Z bar(A)' in X
  3: invokevirtual B.baz:()V;     // method 'void baz()' in B
   ...
\end{bytecode}

Although the above fragment is merely four bytecode instructions
long, it induces several interesting constraints for our phantom
types \code{X}, \code{Y}, and \code{Z}:

\begin{itemize}[--]
\item \code{X} has to support a method \code{bar} accepting an
  argument of type \code{A} and returning a value of type \code{Z}.
\item \code{Y} has to be a subtype of \code{A}, since an actual
  argument of declared type \code{Y} is passed to \code{bar}, which
  has a formal parameter of type \code{A}. This constraint also
  means that if \code{A} is known to be a class (and not an
  interface) then \code{Y} is also a class.
\item \code{Z} has to be a subtype of \code{B}, since a method of
  \code{B} is invoked on an object of declared type \code{Z} (returned
  on top of the stack by the earlier invocation).
\end{itemize}

Our goal is to satisfy all such constraints and
generate definitions of phantom types \code{X}, \code{Y}, and \code{Z}
that are compatible with the bytecode that is available to the tool
(i.e., exists in known classes). Compatibility with existing bytecode
is defined as satisfying the requirements of the Java verifier, which
concern type well-formedness.

Note that such definitions will contain essential parts of missing
structural information for the phantom types: method and field
members, as well as supertypes. Any subsequent static analysis that
will operate on the types produced by complementation will create abstract
objects that are much closer, in structure, to reality.

Of these constraints, the hardest to satisfy are those involving
subtyping.  Constraints on members (e.g., \code{X} has to contain a
``\code{Z bar(A)}'') are easy to satisfy by just adding type-correct
dummy members to the generated classes. This means that the core of
the general program complementation problem is solving the \emph{class
  hierarchy complementation problem}: given a partial type hierarchy
and a set of subtyping constraints, compute a complete type hierarchy
that satisfies the subtyping constraints \emph{without} changing the
direct parents of known types.

Solving the hierarchy complementation problem, constitutes the main
novelty of our approach. The problem appears to be fundamental, and
even of a certain interest in purely graph-theoretic terms. For a
representative special case, consider an object-oriented language with
multiple inheritance (or, equivalently, an interface-only hierarchy in
Java or C\#).  A partial hierarchy, augmented with constraints, can be
represented as a graph, as shown in
Figure~\ref{hiercomp/fig:ex0:problem}. The known part of the hierarchy
is shown as double circles and solid edges. Unknown (i.e., missing)
classes are shown as single circles. Dashed edges represent subtyping
constraints, i.e., indirect subtyping relations that have to hold in
the resulting hierarchy. In graph-theoretic terms, a dashed edge means
that there is a path in the solution between the two endpoints. For
instance, the dashed edge from $C$ to $D$ in
Figure~\ref{hiercomp/fig:ex0:problem} means that the unknown part of
the class hierarchy has a path from $C$ to $D$. This path cannot be a
direct edge from $C$ to $D$, however: $C$ is a known class, so the set
of its supertypes is fixed.

\begin{figure}[ht]
  \begin{minipage}[b]{.5\linewidth}
    \centering
    \includegraphics[scale=0.6]{../figures/complementation/cgraph2.pdf}
    \subcaption{Constraint Graph}\label{hiercomp/fig:ex0:problem}
  \end{minipage}
  \begin{minipage}[b]{.5\linewidth}
    \centering
    \includegraphics[scale=0.6]{../figures/complementation/cgraph2-solution.pdf}
    \subcaption{Solution}\label{hiercomp/fig:ex0:solution}
  \end{minipage}
  \caption[Example of constraints in a multiple inheritance setting]{%
    Example of constraints in a multiple inheritance
    setting. Double-circles signify known classes, single circles
    signify unknown classes. Solid edges (``known edges'') signify
    direct subtyping, dashed edges signify transitive subtyping.}
  \label{hiercomp/fig:ex0}
\end{figure}

In order to solve the above problem instance, we need to compute a
directed acyclic graph (DAG) over the same nodes,\footnote{Inventing
extra nodes does not contribute to a solution in this problem.} so
that it preserves all known nodes and edges, and adds edges \emph{only
to unknown nodes} so that all dashed-edge constraints are
satisfied. That is, the solution will not contain dashed edges
(indirect subtyping relationships), but every dashed edge in the input
will have a matching directed path in the solution
graph. Figure~\ref{hiercomp/fig:ex0:solution} shows one such possible solution.
As can be seen, solving the constraints (or determining that they are
unsatisfiable) is not trivial. In this example, any solution has to
include an edge from $B$ to $E$, for reasons that are not immediately
apparent. Accordingly, if we change the input of
Figure~\ref{hiercomp/fig:ex0:problem} to include an edge from $E$ to $B$, then
the constraints are not satisfiable---any attempted solution
introduces a cycle. The essence of the algorithmic difficulty of the
problem (compared to, say, a simple topological sort) is that we
cannot add extra direct parents to known classes $A$ and $C$---any
subtyping constraints over these types have to be satisfied via
existing parent types. This corresponds directly to our high-level
program requirement: we want to compute definitions for the missing
types only, without changing existing code.
%
For a language with single inheritance, the problem is similar, with
one difference: the solution needs to be a tree instead of a DAG. (Of
course, the input in Figure~\ref{hiercomp/fig:ex0:problem} already violates the
tree property since it contains known nodes with multiple known
parents.)

We provide algorithms to solve the hierarchy complementation problem
in the single inheritance and multiple inheritance settings.  We also
show that the problem in a language such as Java, with single
inheritance but multiple subtyping and distinguished class
vs. interface types, can be decomposed into separate single- and
multiple-subtyping instances.  We implement our algorithms in a tool,
JPhantom,\footnote{JPhantom is available online at
  \url{https://github.com/gbalats/jphantom}} which complements partial
Java bytecode programs so that the result is guaranteed to satisfy the
Java verifier requirements. In a sense, JPhantom aims to recover
structural information for phantom classes, via inference, by tracking
their use in existing code. JPhantom is highly scalable and runs in
mere seconds even for large input applications and complex constraints
(with a maximum of 14s for a 19MB binary).

\section{Conclusions}

To summarize, we advocate that there are many opportunities in
recovering implicit structural information about memory that can
improve static analysis of programs, but require complex inference
that takes advantage of indirect uses of types. We have examined three
different scenarios to test and evaluate our thesis, regarding
\begin{inparablank}
\item generic C/C++ programs, and
\item Java programs that either use reflection or
\item are missing parts of their code.
\end{inparablank}
In all cases, we where able to improve static analysis, by recovering
memory structure that was not previously evident.

\section{Publications}

The contents of this doctoral dissertation are based on the following
published papers:

\begin{itemize}[--]
\item \emph{Structure-Sensitive Points-To Analysis for C and C++}
  \cite{structsens}
\item \emph{More Sound Static Handling of Java Reflection}
  \cite{reflection}
\item \emph{Class Hierarchy Complementation: Soundly Completing a
    Partial Type Graph}~\cite{jphantom}
\item \emph{Pointer Analysis} \cite{survey}
\end{itemize}


% \clearpage
\bibliographystyle{splncs03}
\bibliography{../bib/my-publications,../bib/thesis,../bib/class-hier-comp,../bib/reflection,../bib/ptr-analysis/ptranalysis,../bib/ptr-analysis/specs,../bib/ptr-analysis/tools,../bib/ptr-analysis/proceedings}

\end{document}
