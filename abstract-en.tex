%%% Abstract in English

Static analysis aims to achieve a deep understanding of program
behavior, by automatic reasoning that requires only the program's
source code and not any actual execution. To reach a truly deep level
of program understanding, static analysis techniques need to create an
abstraction of memory that covers all possible executions. Such
abstract models may quickly degenerate after losing essential
structural information about the memory objects they describe, due to
the use of specific programming idioms and language features, or
because of practical analysis limitations. In many cases, some of the
lost memory structure may be retrieved, though it requires complex
inference that takes advantage of indirect uses of types. Such
recovered structural information may, then, greatly benefit static
analysis.

This dissertation shows how we can recover structural information,
first, in the context of C/C++, by introducing a structure-sensitive
pointer analysis that refines its abstraction based on type
information that is discovered on-they-fly. Next, in the context of
higher-level languages without direct memory access, like Java, we
identify two primary causes of losing structural information:
\begin{inparaenum}[(i)]
\item the use of reflection, and
\item analysis of partial programs.
\end{inparaenum}
We present techniques that extend a standard Java pointer analysis by
building on top of state-of-the-art handling of reflection. The
principle is similar to our structure-sensitive analysis for C/C++: to
track the use of reflective objects, \emph{during} pointer analysis,
to gain important insights on their structure, which can be used to
``patch'' the handling of reflective operations on the running
analysis, in a mutually recursive fashion.

%%% Local Variables:
%%% mode: latex
%%% TeX-master: "thesis"
%%% End:
