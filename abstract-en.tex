%%% Abstract in English

Static analysis aims to achieve an understanding of program behavior,
by means of automatic reasoning that requires only the program's
source code and not any actual execution. To reach a truly broad level
of program understanding, static analysis techniques need to create an
abstraction of memory that covers all possible executions. Such
abstract models may quickly degenerate after losing essential
structural information about the memory objects they describe, due to
the use of specific programming idioms and language features, or
because of practical analysis limitations. In many cases, some of the
lost memory structure may be retrieved, though it requires complex
inference that takes advantage of indirect uses of types. Such
recovered structural information may, then, greatly benefit static
analysis.

This dissertation shows how we can recover structural information,
first, in the context of C/C++, by introducing a structure-sensitive
pointer analysis that refines its abstraction based on type
information that it discovers on-they-fly. This analysis is
implemented in \cclyzer{}, a static analysis tool for LLVM bitcode.
%
Next, in the context of higher-level languages without direct memory
access, like Java, we identify two primary causes of losing structural
information:
\begin{inparaenum}[(i)]
\item the use of reflection, and
\item analysis of partial programs.
\end{inparaenum}
We present techniques that extend a standard Java pointer analysis by
building on top of state-of-the-art handling of reflection. The
principle is similar to that of our structure-sensitive analysis for
C/C++: track the use of reflective objects, \emph{during} pointer
analysis, to gain important insights on their structure, which can be
used to ``patch'' the handling of reflective operations on the running
analysis, in a mutually recursive fashion.
%
To address the challenge of analyzing partial Java programs in full
generality, we define the problem of ``program complementation'':
given a partial program we seek to provide definitions for its missing
parts so that the ``complement'' satisfies all static and dynamic
typing requirements induced by the code under analysis. Essentially,
complementation aims to recover the structure of phantom types.  Apart
from discovering missing class members (i.e., fields and methods),
satisfying the subtyping constraints leads to the formulation of a
novel typing problem in the OO context, regarding type hierarchy
complementation. We offer algorithms to solve this problem in various
inheritance settings, and implement them in JPhantom, a practical tool
for Java bytecode complementation.
%
Finally, we show that, in all cases, the recovered structural
information greatly benefits static analysis on the program.

%%% Local Variables:
%%% mode: latex
%%% TeX-master: "thesis"
%%% End:
